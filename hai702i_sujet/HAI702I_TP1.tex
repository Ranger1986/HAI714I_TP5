\documentclass[a4paper,10pt]{tp_um}


\makeatletter
%--------------------------------------------------------------------------------

%\usepackage{dsfont}
\usepackage[utf8]{inputenc}
\usepackage{lmodern}
\usepackage{dsfont}
\usepackage[T1]{fontenc}
\usepackage[english,french]{babel}

\usepackage{qrcode}

\usepackage{eurosym}
\usepackage{diagbox}
\usepackage{enumitem}\setlist{nosep}\def\labelitemi{--}

\usepackage{graphicx}
\usepackage{float}

\usepackage{array}
\newcolumntype{L}[1]{>{\raggedright\let\newline\\\arraybackslash\hspace{0pt}}m{#1}}
\newcolumntype{C}[1]{>{\centering\let\newline\\\arraybackslash\hspace{0pt}}m{#1}}
\newcolumntype{R}[1]{>{\raggedleft\let\newline\\\arraybackslash\hspace{0pt}}m{#1}}
%----------
%  Version
%-----------

\usepackage{fancyhdr} 
\usepackage{minted}

%--------
%  Tkz  
%--------
\usepackage{blkarray}
\usepackage[babel=true,kerning=true]{microtype}
\usepackage[caption=false]{subfig}
\usepackage{xcolor,colortbl}
\usepackage{diagbox,calc,soul,graphicx}

\usepackage{tikz}
\usetikzlibrary{3d,calc,fadings,decorations.pathreplacing,matrix,arrows,decorations.text}
\usetikzlibrary{patterns}
\usetikzlibrary{positioning}
\usetikzlibrary{babel}
\usetikzlibrary{shapes}
\usetikzlibrary{shadings}
\usetikzlibrary{cd}
\usepackage{tikz-3dplot}
\usepackage{pgfplots}
\usepgfplotslibrary{fillbetween}
\pgfplotsset{compat=newest}
\usepgfplotslibrary{external} 
\tikzexternalize[prefix=output_latex/]
\usepgfplotslibrary{fillbetween}
%\graphicspath{{./output-latex/}}


%\usepackage{etex}
%\reserveinserts{28}
%\usepackage{pstricks}
%\usepackage{pst-solides3d}

\usepackage{gnuplot-lua-tikz}

\newcommand\chideux[1]{#1<=0||(#1!=int(#1))?1/0:x<=0?0.0:exp((0.5*#1-1.0)*log(x)-0.5*x-lgamma(0.5*#1)-#1*0.5*log(2))}
\newcommand\gauss[2]{1/(#2*sqrt(2*pi))*exp(-((x-#1)^2)/(2*#2^2))} 
\newcommand\student[1]{gamma(.5*(#1+1))/(sqrt(#1*pi)*gamma(.5*#1))*((1+x^2/#1)^(-.5*(#1+1)))}

% display dices
\usepackage{xparse}\usetikzlibrary{shapes}
\NewDocumentCommand{\drawdie}{O{}m}{%
\begin{tikzpicture}[x=1em,y=1em,radius=0.1,#1,baseline=0.575ex]
		\draw[rounded corners=0.5] (0,0) rectangle (1,1);
	  \ifodd#2
      \fill[] (0.5,0.5) circle;
        \fi
  \ifnum#2>1
      \fill[] (0.2,0.2) circle;
          \fill[] (0.8,0.8) circle;
     \ifnum#2>3
          \fill[] (0.2,0.8) circle;
       \fill[] (0.8,0.2) circle;
           \ifnum#2>5
         \fill[] (0.8,0.5) circle;
       \fill[] (0.2,0.5) circle;
            \ifnum#2>7
           \fill[] (0.5,0.8) circle;
          \fill[] (0.5,0.2) circle;
        \fi
    \fi
      \fi
      \fi
\end{tikzpicture}%
} 
%------------------
% Math environment
%------------------

\usepackage{latexsym}
\usepackage{amsmath}
\usepackage{amsbsy}
\usepackage{amsfonts}
\usepackage{amssymb}
\usepackage{nicefrac}
\usepackage{amscd}
\usepackage{amsthm}
\usepackage{mathtools}

\newtheoremstyle{definitionSs}{\topsep}{\topsep}%
     {}%         Body font
     {}%         Indent amount (empty = no indent, \parindent = para indent)
     {\sffamily\bfseries}% Thm head font
     {.}%        Punctuation after thm head
     { }%     Space after thm head (\newline = linebreak)
     {\thmname{#1}\thmnumber{~#2}\thmnote{~#3}}%         Thm head spec

\newtheoremstyle{plainSs}{\topsep}{\topsep}%
     {\itshape}%         Body font
     {}%         Indent amount (empty = no indent, \parindent = para indent)
     {\sffamily\bfseries}% Thm head font
     {.}%        Punctuation after thm head
     { }%     Space after thm head (\newline = linebreak)
     {\thmname{#1}\thmnumber{~#2}\thmnote{~#3}}%         Thm head spec

\theoremstyle{definitionSs}
\newtheorem{remark}{Remarque}
%\newtheorem{definition}{Définition}[section]
%\newtheorem{app}{Application}[section]
\newtheorem{exemple}{Exemple}[section]
%\newtheorem{exo}{Exercice}[section]
%\newtheorem{proposition}{Proposition}[section]
%\newtheorem{lemme}{Lemme}[section]
%\newtheorem{theorem}{Théorème}[section]
%\newtheorem{defprop}{D\'efinition-Proposition}[section]


%\usepackage[framemethod=tikz]{mdframed}
\usepackage[]{mdframed}

\newmdtheoremenv[
hidealllines=true,
leftline=true,
skipabove=0pt,
innertopmargin=-5pt,
innerbottommargin=2pt,
linewidth=4pt,
linecolor=gray!90,
innerrightmargin=0pt,
]{definition}{Définition}[section]

\newmdtheoremenv[
hidealllines=true,
leftline=true,
skipabove=0pt,
innertopmargin=-5pt,
innerbottommargin=2pt,
linewidth=4pt,
linecolor=gray!40,
innerrightmargin=0pt,
]{proposition}{Proposition}[section]

\newmdtheoremenv[
hidealllines=true,
leftline=true,
skipabove=0pt,
innertopmargin=-5pt,
innerbottommargin=2pt,
linewidth=4pt,
linecolor=gray!40,
innerrightmargin=0pt,
]{corollaire}{Corollaire}[section]

\newmdtheoremenv[
hidealllines=true,
leftline=true,
skipabove=0pt,
innertopmargin=-5pt,
innerbottommargin=2pt,
linewidth=4pt,
linecolor=gray!90,
innerrightmargin=0pt,
]{defprop}{Définition - Proposition}[section]

\newmdtheoremenv[
hidealllines=true,
leftline=true,
skipabove=0pt,
innertopmargin=-5pt,
innerbottommargin=2pt,
linewidth=4pt,
linecolor=gray!100,
innerrightmargin=0pt,
]{theorem}{Théorème}[section]

%---------------
% Mise en page
%--------------

\setlength{\parindent}{0pt}

\renewcommand*{\descriptionlabel}[1]{\hspace\labelsep{\itshape #1}}
\renewcommand{\emph}[1]{{\sffamily\bfseries #1}}

%\usepackage{titlesec, blindtext, color}
%\definecolor{gray75}{gray}{0.75}
\definecolor{dgreen}{RGB}{0,100,0}
%\newcommand{\hsp}{\hspace{20pt}}
%\titleformat{\chapter}[hang]{\sffamily\Huge\bfseries}{\thechapter\hsp\textcolor{gray75}{|}\hsp}{0pt}{\Huge\bfseries}
\usepackage{sectsty}
\usepackage{xcolor}
\definecolor{astral}{RGB}{46,116,181}
\allsectionsfont{\color{astral}\normalfont\sffamily\bfseries}

\usepackage[subfigure]{tocloft}
\renewcommand{\cftchapfont}{\sffamily\bfseries\color{astral}}
\renewcommand{\cfttoctitlefont}{\sffamily\bfseries\Huge\color{astral}}
\usepackage{hyperref}

% Poly pour étudiants
%\newif\ifteacherVersion
%\teacherVersiontrue % comment out to hide answers
%\ifteacherversion
%Answer
%\fi
\providecommand{\blanc}[1]{\vspace*{#1}}

%\usepackage{tocloft}
%\renewcommand{\cftchapfont}{\sffamily\bfseries}
%----------------
% Some commands
%----------------
\makeatletter
\newcommand\RedeclareMathOperator{%
  \@ifstar{\def\rmo@s{m}\rmo@redeclare}{\def\rmo@s{o}\rmo@redeclare}%
}
% this is taken from \renew@command
\newcommand\rmo@redeclare[2]{%
  \begingroup \escapechar\m@ne\xdef\@gtempa{{\string#1}}\endgroup
  \expandafter\@ifundefined\@gtempa
     {\@latex@error{\noexpand#1undefined}\@ehc}%
     \relax
  \expandafter\rmo@declmathop\rmo@s{#1}{#2}}
% This is just \@declmathop without \@ifdefinable
\newcommand\rmo@declmathop[3]{%
  \DeclareRobustCommand{#2}{\qopname\newmcodes@#1{#3}}%
}
\@onlypreamble\RedeclareMathOperator
\makeatother

\DeclareMathOperator{\aire}{Aire}
\providecommand{\gf}{g\circ f}
\providecommand{\R}{\ensuremath \mathbb{R}}
\providecommand{\reg}[1]{\mathcal{C}^{#1}}
\providecommand{\1}{\mathbb{1}}
\providecommand{\N}{\mathbb{N}}
\providecommand{\M}{\mathcal{M}}
\providecommand{\Q}{\mathbb{Q}}
\renewcommand{\L}{\mathcal{L}}
\providecommand{\D}{\mathcal{D}}
\providecommand{\Cc}{\mathcal{C}}
\providecommand{\F}{\mathcal{F}}
\providecommand{\Ee}{\mathcal{E}}
\providecommand{\G}{\mathcal{G}}
\providecommand{\Z}{\mathbb{Z}}
\providecommand{\x}{\ensuremath\boldsymbol{x}}
\providecommand{\y}{\ensuremath\boldsymbol{y}}
\providecommand{\1}{\mathbb{1}}
\providecommand{\p}{\partial}
\providecommand{\Pp}{\mathcal{P}}
\providecommand{\P}{\mathbb{P}}
\providecommand{\E}{\mathbb{E}}
\providecommand{\U}{\mathcal{U}}
\providecommand{\V}{\mathcal{V}}
\providecommand{\ie}{\textit{i.e. }}
\renewcommand{\P}{\mathbb{P}}
\renewcommand{\S}{\mathcal{S}}
\providecommand{\E}{\mathbb{E}}
\providecommand{\one}{\mathds{1}}
\DeclareMathOperator{\card}{Card}
\DeclareMathOperator{\vol}{Vol}
\DeclareMathOperator{\var}{Var}
\DeclareMathOperator{\vect}{\mathsf{Vect}}
\DeclareMathOperator{\med}{median}
\DeclareMathOperator{\hess}{Hess}
\DeclareMathOperator{\jac}{Jac}
\DeclareMathOperator{\cov}{cov}
\DeclareMathOperator{\im}{\mathsf{Im}}
\RedeclareMathOperator{\ker}{\mathsf{Ker}}
\RedeclareMathOperator{\det}{\mathsf{det}}
\DeclareMathOperator{\id}{\mathsf{Id}}
\DeclareMathOperator{\can}{\mathsf{can}}
\DeclareMathOperator{\com}{\mathsf{com}}

\providecommand{\ncd}{\norm{\cdot}}
\providecommand{\norm}[1]{\left\lVert#1\right\rVert}
\providecommand{\bnorm}[1]{\bigg\lVert#1\bigg\rVert}
\providecommand{\snorm}[1]{\lVert#1\rVert}

\newcommand{\tnorm}[1]{{\left\vert\kern-0.25ex\left\vert\kern-0.25ex\left\vert #1 
    \right\vert\kern-0.25ex\right\vert\kern-0.25ex\right\vert}}

\providecommand{\abs}[1]{\left\lvert#1\right\rvert}
\providecommand{\sabs}[1]{\lvert#1\rvert}
\providecommand{\babs}[1]{\bigg\lvert#1\bigg\rvert}

\providecommand{\prscd}{\prs{\cdot,\cdot}}
\providecommand{\prs}[1]{\left\langle #1\right\rangle}
\providecommand{\sprs}[1]{\langle #1\rangle}
\providecommand{\bprs}[1]{\bigg\langle #1\bigg\rangle}

\providecommand{\rev}{$\R$ espace vectoriel}

\providecommand{\dpar}[2]{\frac{\partial #1}{\partial #2}}

% Multiversioning 


\usepackage{ifthen}

\newcommand{\pl}[1]{%
	\ifthenelse{\equal{\version}{poly}}{#1}{}%
	\ifthenelse{\equal{\version}{polyProf}}{#1}{}%
}
\newcommand{\plprof}[1]{\ifthenelse{\equal{\version}{polyProf}}{#1}{}}
\newcommand{\sld}[1]{\ifthenelse{\equal{\version}{slide}}{#1}{}}


\mdfdefinestyle{response}{
	leftmargin=.01\textwidth,
	rightmargin=.01\textwidth,
	linewidth=1pt
	hidealllines=false,
	leftline=true,
	rightline=true,topline=true,bottomline=true,
        skipabove=\baselineskip,%0pt,
	%innertopmargin=-5pt,
	%innerbottommargin=2pt,
	linecolor=black,
	innerrightmargin=0pt,
	}
%\providecommand{\rep}[1]{$ $ \newline \begin{mdframed}[style=response] \vspace*{#1} \end{mdframed}}

% generate breakable white space allowing students to write notes.
\newcommand*{\DivideLengths}[2]{%
  \strip@pt\dimexpr\number\numexpr\number\dimexpr#1\relax*65536/\number\dimexpr#2\relax\relax sp\relax
}

\providecommand{\rep}[1]{$ $

    \begin{mdframed}[style=response]  
	\vspace*{\DivideLengths{#1}{3cm}cm}
	\pagebreak[1]	
	\vspace*{\DivideLengths{#1}{3cm}cm}
	\pagebreak[1]		
	\vspace*{\DivideLengths{#1}{3cm}cm}
    \end{mdframed}%
}
\providecommand{\repcom}[1]{\begin{mdframed}[style=response] #1 \end{mdframed}}

\def\redspace{\sld{\setlength{\belowdisplayskip}{0pt} \setlength{\belowdisplayshortskip}{0pt}\setlength{\abovedisplayskip}{0pt}\setlength{\abovedisplayshortskip}{0pt}}}
%------------------------------------------------------------------------------
%\DeclareUnicodeCharacter{00A0}{~}

\pdfstringdefDisableCommands{%
    %\renewcommand*{\bm}[1]{#1}%
    \renewcommand*{\R}{R}%
    % any other necessary redefinitions 
}
\makeatother




\newcommand\code[1]{\mintinline{cpp}{#1}}

\title{\Large \sffamily\bfseries TP 1}
\ue{HAI702I}

\begin{document}

\maketitle

\bigskip
\bigskip
\bigskip

Au d\'ebut de ce TP/TD, vous recevrez une archive zip contenant une base de code.
Ce code permet d'afficher un \emph{maillage triangulaire} \`a l'aide d'\texttt{openGL}.
\begin{enumerate}
 \item Vous trouverez la description du code dans le sujet.
 \item Vous devez faire \'evoluer ce code au fur et \`a mesure du TP, pour r\'epondre aux questions.
 \item Vous devez déposez votre code, fichier image de résultat et une phrase sur l'état de votre TP sur le moodle.
\end{enumerate}


\section{Base de code}

T\'el\'echargez l'archive sur le moodle \url{https://moodle.umontpellier.fr/course/view.php?id=22845}. Pour compiler le code et l'exécuter: 
\begin{minted}[bgcolor=blue!4,fontsize=\footnotesize,mathescape,tabsize=4,escapeinside=||, breaklines]{bash}
$ make
$ ./tp
\end{minted}

%TODO Faire la description du code
\subsection{Interactions utilisateur}
\begin{minted}[bgcolor=blue!4,fontsize=\footnotesize,mathescape,tabsize=4,escapeinside=||, linenos, breaklines]{cpp}
void key (unsigned char keyPressed, int x, int y)
\end{minted}
La fonction \code{key} permet de d'interpreter les entrée clavier utilisateur.
Les options de visualisation activée par des touches sont les suivantes, en appuyant sur la touche :
\begin{itemize}
 \item \code{n} : activation/desactivation de l'affichage des normales,
 \item \code{1} : activation/desactivation de l'affichage du modèle d'entrée sur lequel vous effectuez les calculs de normale,
 \item \code{2} : activation/desactivation de l'affichage du modèle transformé,
 \item \code{s} : changement entre l'affichage avec les normales de face et de sommet (maillage plus lisse),
 \item \code{w} : activation/desactivation de l'affichage en fil de fer,
 \item \code{b} : activation/desactivation de l'affichage des repères,
 \item \code{f} : activation/desactivation du mode plein écran.
\end{itemize}

Vous pouvez interagir avec le modèle avec la souris :
\begin{itemize}
 \item Bouton du milieu appuyé : zoomer ou reculer la caméra,
 \item Clique gauche appuyé : faire tourner le modèle.
\end{itemize}

 \subsection{Rendu de maillages}
 
Le fichier tp.cpp contient une m\'ethode \code{draw} est une fonction appelée pour rafraichir l'affichage dès que nécessaire. Elle permet de définir quels sont les élements à afficher, leurs couleurs (\code{glColor3f(r,g,b)} en float de 0 à 1 donnant la couleur RGB = (r*255, g*255, b*255) ). La fonction \code{drawMesh()} permet d'afficher un maillage triangulaire.
Celle-ci fait appel à deux fonctions contenant du code OpenGL basique pour afficher des maillages. 
La première permet un affichage en utilisant les normales au triangles : 
\begin{minted}[bgcolor=blue!4,fontsize=\footnotesize,mathescape,tabsize=4,escapeinside=||, linenos, breaklines]{cpp}
    void drawTriangleMesh( Mesh const & i_mesh ) {
        
        glBegin(GL_TRIANGLES);
        //Iterer sur les triangles
        for(unsigned int tIt = 0 ; tIt < i_mesh.triangles.size(); ++tIt) {
            //Récupération des positions des 3 sommets du triangle pour l'affichage
            //Vertices --> liste indexée de sommets 
            //i_mesh.triangles[tIt][i] --> indice du sommet vi du triangle dans la liste de sommet 
            //pi --> position du sommet vi du triangle
            
            //Normal au triangle
            Vec3 n = i_mesh.triangle_normals[tIt];

            glNormal3f( n[0] , n[1] , n[2] );

            glVertex3f( p0[0] , p0[1] , p0[2] );
            glVertex3f( p1[0] , p1[1] , p1[2] );
            glVertex3f( p2[0] , p2[1] , p2[2] );
        }
        glEnd();
    }
 \end{minted}
La première permet un affichage en utilisant les normales aux sommets :  
\begin{minted}[bgcolor=blue!4,fontsize=\footnotesize,mathescape,tabsize=4,escapeinside=||, linenos, breaklines]{cpp}
    void drawSmoothTriangleMesh( Mesh const & i_mesh ) {
        
        glBegin(GL_TRIANGLES);
        //Iterer sur les triangles
        for(unsigned int tIt = 0 ; tIt < i_mesh.triangles.size(); ++tIt) {
            //Récupération des positions des 3 sommets du triangle pour l'affichage
            //Vertices --> liste indexée de sommets 
            //i_mesh.triangles[tIt][i] --> indice du sommet vi du triangle dans la liste de sommet 
            //pi --> position du sommet vi du triangle
            //ni --> normal du sommet vi du triangle pour un affichage lisse
            Vec3 p0 = i_mesh.vertices[i_mesh.triangles[tIt][0]];
            Vec3 n0 = i_mesh.normals[i_mesh.triangles[tIt][0]];

            Vec3 p1 = i_mesh.vertices[i_mesh.triangles[tIt][1]];
            Vec3 n1 = i_mesh.normals[i_mesh.triangles[tIt][1]];

            Vec3 p2 = i_mesh.vertices[i_mesh.triangles[tIt][2]];
            Vec3 n2 = i_mesh.normals[i_mesh.triangles[tIt][2]];

            //Passage des positions et normales à OpenGL
            glNormal3f( n0[0] , n0[1] , n0[2] );
            glVertex3f( p0[0] , p0[1] , p0[2] );
            glNormal3f( n1[0] , n1[1] , n1[2] );
            glVertex3f( p1[0] , p1[1] , p1[2] );
            glNormal3f( n2[0] , n2[1] , n2[2] );
            glVertex3f( p2[0] , p2[1] , p2[2] );
    }
    glEnd();
    }
 \end{minted}
\paragraph{Remarque}
Vous pouvez vous inspirer de ce code pour le parcours du maillage nécessaire au calcul des normales. 

Vous remarquerez qu'il y a plusieurs fonction commençant par \code{draw}, elles permettant l'affiche de vecteurs, repères et champs de vecteurs tels que les normales.
\section{Calcul géométrique et implémentation de classes}

\label{sec_exo_Vec3}

Compléter la classe \code{Vec3} qui contient les fonctions essentielles pour le calcul de base : assignation, somme, soustraction, multiplication et division par un scalaire, produit scalaire, produit vectoriel, norme,\ldots

\begin{minted}[bgcolor=blue!4,fontsize=\footnotesize,mathescape,tabsize=4,escapeinside=||, linenos, breaklines]{cpp}
    class Vec3 {
private:
    float mVals[3];
public:
    Vec3() {}
    Vec3( float x , float y , float z ) {
       mVals[0] = x; mVals[1] = y; mVals[2] = z;
    }
    float & operator [] (unsigned int c) { return mVals[c]; }
    float operator [] (unsigned int c) const { return mVals[c]; }
    void operator = (Vec3 const & other) {
       mVals[0] = other[0] ; mVals[1] = other[1]; mVals[2] = other[2];
    }
    float squareLength() const {
       return mVals[0]*mVals[0] + mVals[1]*mVals[1] + mVals[2]*mVals[2];
    }
    float length() const { return sqrt( squareLength() ); }
    void normalize() { float L = length(); mVals[0] /= L; mVals[1] /= L; mVals[2] /= L; }
    static float dot( Vec3 const & a , Vec3 const & b ) {
       //TODO
    }
    static Vec3 cross( Vec3 const & a , Vec3 const & b ) {
        //TODO
    }
    void operator += (Vec3 const & other) {
        mVals[0] += other[0];
        mVals[1] += other[1];
        mVals[2] += other[2];
    }
    void operator -= (Vec3 const & other) {
        mVals[0] -= other[0];
        mVals[1] -= other[1];
        mVals[2] -= other[2];
    }
    void operator *= (float s) {
        mVals[0] *= s;
        mVals[1] *= s;
        mVals[2] *= s;
    }
    void operator /= (float s) {
        mVals[0] /= s;
        mVals[1] /= s;
        mVals[2] /= s;
    }
};
\end{minted}

\begin{enumerate}
 \item Compléter les fonctions de calcul du produit scalaire \code{dot} et du produit vectoriel \code{cross}
\item Compléter la fonctions de \code{Mat3} permettant d'effectuer un produit matriciel
\item Compléter la fonction de \code{Mat3} permettant d'effectuer la multiplication d'appliquer une matrice de transformation à un point : \code{Mat3*Vec3}.
\end{enumerate}


\begin{remark}
    \begin{enumerate}
        \item On a revu les notions associées au produit scalaire, notamment :
            \begin{itemize}
                \item Le produit scalaire entre le vecteur $\mathbf{a}$ et le vecteur $\mathbf{b}$ donne le produit des longueurs multiplié par le cosinus de
                    l’angle formé par ces vecteurs.
                \item Le produit scalaire entre un vecteur $\mathbf{a}$ et un vecteur unitaire $\mathbf{d}$ vaut la longueur (attention au signe !) de $\mathbf{a}$ projeté
                    sur la (demi-)droite dirigée par $\mathbf{d}$.
            \end{itemize}

        \item On a revu les notions associées au produit vectoriel, notamment :
            \begin{itemize}
                \item Le produit vectoriel entre le vecteur $\mathbf{a}$ et le vecteur $\mathbf{b}$ est orthognal à ces deux vecteurs (et sa direction est donnée
                        par la ``règle de la main droite''.
                    \item Sa norme vaut le produit des longueurs multiplié par le sinus de l’angle formé par ces vecteurs.
                \end{itemize}
        \end{enumerate}
    \end{remark}

\section{Application}

\begin{enumerate}
\item Pour tester vos calculs, compléter les fonctions de calcul de normales aux faces triangulaires \code{computeTrianglesNormals()}. Les normales calculées seront celles du modèle vert. Vous pourrez comparer vos résultats avec les normales du maillage gris \code{transformed_mesh}. 
\item Calculez ensuite les normales par sommet en faisant la moyenne des normales des triangles incidents. N'oubliez pas de normaliser.
\item Remplacer 
\begin{minted}[bgcolor=blue!4,fontsize=\footnotesize,mathescape,tabsize=4,escapeinside=||, linenos, breaklines]{cpp}
    for( unsigned int i = 0 ; i < mesh.vertices.size() ; ++i ) {
        transformed_mesh.vertices.push_back( mesh.vertices[i] + mesh_transformation.translation );
    }
\end{minted}
par 
\begin{minted}[bgcolor=blue!4,fontsize=\footnotesize,mathescape,tabsize=4,escapeinside=||, linenos, breaklines]{cpp}
    for( unsigned int i = 0 ; i < mesh.vertices.size() ; ++i ) {
        transformed_mesh.vertices.push_back( mesh_transformation.rotation*mesh.vertices[i] 
        + mesh_transformation.translation );
    }
\end{minted}

\item Essayer differentes transformation en mettant à jour  \code{mesh_transformation.rotation} et \code{mesh_transformation.translation}. 
Créer une matrice \code{scale} de mise à l'échelle non-uniforme et essayer \code{mesh_transformation.rotation = mesh_transformation.rotation*scale}. Regarder les normales, que constatez vous?

\end{enumerate}

%\section{La STL (Standard Template Library)}


%\begin{itemize}
 %\item Un \code{vector} est un tableau dont l’accès aux éléments est immédiat, et %que l’on peut redimensionner.
%\item Une \code{list} est un ensemble auquel on peut ajouter ou retirer %facilement des éléments en bout de chaîne.
%\item Une \code{map} est une table qui contient des couples clé/valeur.
%\end{itemize}








\end{document}
